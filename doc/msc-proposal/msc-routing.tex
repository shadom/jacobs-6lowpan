\chapter{6LoWPAN Routing}\label{ch:routing}
Neither the IEEE 802.15.4 \cite{ieee802.15.4} standard nor the 6LoWPAN format specification \cite{rfc4944} define how mesh topologies could be obtained and maintained. In 6LoWPAN routing can be performed either in the IP-layer, using a Route Over approach, or in the adaptation layer, described in Section \ref{subsec:mesh.header}, using the Mesh Under approach. The Mesh Under approach performs the multi-hop communication below the IP link and therefore a 6LoWPAN is seen as a single IP link. 

A 6LoWPAN network comprises nodes which are either hosts or routers. A host is a node that only sources or sinks IPv6 datagrams, whereas a router forwards datagrams between arbitrary source-destination pairs. Routers are present only in a Route Over configuration, where the network is composed of overlapping link-local scopes, to overcome non-transitive nature of links. 
