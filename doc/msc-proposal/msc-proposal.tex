\documentclass[12pt, titlepage, a4paper]{report}
% For printed document - always starts a new chapter on the right-side page (leaves blank pages):
%\documentclass[12pt, titlepage, a4paper, openright, twoside]{report}

%% Needed packages:
\usepackage{pdfpages} 		% used for introducing the title PDF page
%\usepackage{a4wide}		% spreads text over the whole width of the page
\usepackage[utf8x]{inputenc}	% encoding
\usepackage[english]{babel}	% multilingual typesetting
\usepackage{graphicx}		% adds graphics
\usepackage[pdftex]{hyperref}	% hyper-refs inside the doc
\usepackage{url}		% formats urls
\usepackage{cite}		% citations from bibtex
\usepackage{makeidx}		% index
\usepackage{robustindex}	% index

%% Add/remove packages here:
\usepackage{ucs}
\usepackage{parskip}
\usepackage{wrapfig}
\usepackage{array}
\usepackage{rotating}
\usepackage{alltt}
\usepackage{latexsym}
\usepackage{amsmath, amsthm, amssymb}
\usepackage{setspace} 
%\usepackage[hide]{ed}
\usepackage{color}
\usepackage{listings}
\lstset{float=htb,columns=fullflexible,frame=lines,basicstyle=\scriptsize, language=XML,
        numbers=left,stepnumber=5,numberstyle=\tiny,showstringspaces=false}

\pagestyle{headings}
\makeindex    % obligatory with robustindex

%%%%%%%%%%%%%%%%%%%%%%%%%%%%%%%%%%%%%%%%%%%%%%%%%%%%%%%%%%%%%%%%%%%%%%%%%%%%%%%%
%% Document
\begin{document}

\begin{titlepage}
\includepdf[pages=1]{title}
\thispagestyle{empty}
\end{titlepage}

%% Acknowledgements page
\newpage
\thispagestyle{empty}

%% Abstract page
\newpage
\begin{abstract}
Abstract
\end{abstract}

\newpage
\tableofcontents
%\listoffigures 
\newpage 

%%
%	Introduction
%%
\chapter{Introduction}
* LowPAN, possible applications (use cases), why it's important.\\
* The purpose of the proposed work will be to investigate smth. The proposal describes problems associated with enabling IP communication with devices in a LoWPAN.\\
* As preparation for this work, several related areas have been studied in more
depth and the collected background information; Structure of the proposal.\\

\chapter{LoWPAN Networks}
Low-power and lossy networks (LLNs) are the networks made of highly constrained nodes (limited CPU, memory, power) interconnected by a variety of "lossy" links (low-power radio links). Such networks can be characterized by low speed, low performance, low cost, and unstable connectivity.  A low-power wireless personal area network (LoWPAN) is a particular instance of an LLN, comprised devices that conform to the IEEE 802.15.4 standard \cite{ieee802.15.4}.

%%
%	Transmission of IPv6 Packets
%%
\chapter{IPv6 over LoWPAN Networks}

%%
%	Conclusion
%%
\chapter*{Conclusion}
\addcontentsline{toc}{chapter}{Conclusion}

%% Bibliography
\newpage
\bibliographystyle{plain}
\bibliography{msc-proposal}
\addcontentsline{toc}{chapter}{Bibliography}

\nocite{ieee802.15.4} 
\nocite{rfc2460} 
\nocite{rfc4291} 
\nocite{rfc4443}
\nocite{rfc4861}
\nocite{rfc4862}
\nocite{rfc4919}
\nocite{rfc4944} 
\nocite{draft-usecases-05} 
\nocite{draft-hc-06} 
\nocite{draft-nd-06}

\end{document}
