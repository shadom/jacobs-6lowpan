\chapter{6LoWPAN Neighbor Discovery Protocol}\label{ch:nd}
The Neighbor Discovery (ND) protocol for IPv6 \cite{rfc4861} provides for basic bootstrapping and network operation. Nodes use ND to determine the link-layer addresses for neighbors known to reside on attached links, as well as to find neighboring routers that are willing to forward packets. However, the standard IPv6 Neighbor Discovery has several problems when using it with 6LoWPANs. IPv6 ND heavily uses multicast capabilities, whereas it is very expensive and not desirable in a low-power, lossy wireless network. Moreover, in a Route Over configuration LoWPAN links are non-transitive and composed of a large amount of overlapping radio ranges, but the classic ND was not designed for such links. Therefore, the classic ND is not suitable for 6LoWPANs. 

The Internet Draft \cite{draft-nd-07} specifies an optimized neighbor discovery mechanism sufficient for LoWPAN operation. The 6LoWPAN Neighbor Discovery (6LoWPAN-ND) introduces a node registration mechanism optimizing the node-router interface, which requires no flooding and reduces link-local multicast frequency.  The concept of a LoWPAN Whiteboard located at Edge Routers (routers that connect a LoWPAN to another IP network) is introduced, which allows for Duplicate Address Detection for the entire LoWPAN. The solution supports both Mesh Under and Route Over configurations for multihop forwarding. This chapter describes the specified 6LoWPAN-ND protocol.

%All nodes in the LoWPAN register with Routers and Edge Routers, though Routers are present only in a Route Over configuration. Each Router maintains a set of information about nodes that are currently registered through it, called the binding table. All IPv6 addresses in the LoWPAN are stored within a conceptual data structure, Whiteboard, located at Edge Routers (ERs). Nodes send periodic registration messages in order to maintain their bindings in the Router binding tables and Edge Router Whiteboard. 

\section{Bootstrapping and Basic Operation}
Bootstrapping of a LoWPAN node consists of several steps.  At first, a node is required to autoconfigure at least one address, a link-local address, which is derived from the IEEE 64-bit extended MAC. In order to receive RAs from routers, a node joins the all-nodes multicast address and, if the node is a router, the all-routers multicast address. Once the interfaces have been initialized, a node listens for Router Advertisements (RA) from Edge Routers or LoWPAN Routers, or broadcasts a Router Solicitation (RS). Upon receipt of the RA, the node forms an optimistic global unique address with stateless address autoconfiguration and chooses one or more default routers (routers are present only in a Route Over configuration). 

The constructed global address is tentative as long as the binding is not confirmed by the ER. To accomplish this, the node performs initial registration with the ER by sending a unicast Node Registration (NR) message with a list of all addresses it wants to register. The ER replies with a Node Confirmation(NC), which includes the set of addresses confirmed to be bound to the Whiteboard of the ER. Once the node has received the NC message, it is capable to send packets to any IPv6 address inside or outside the LoWPAN. Each router maintains a set of information about nodes that are currently registered through it, called the binding table. Nodes send periodic registration messages in order to maintain their bindings in the router binding tables and Edge Router Whiteboard. The detection of duplicate addresses (DAD) is also performed as part of the node registration process by the ER using a lookup on the Whiteboard. 

This information about link-local addresses is collected during the node registration process. Nodes store the information about router link-local addresses in their default router list, while routers keep the information about nodes in their binding tables. 



\section{Message Formats}

6LoWPAN-ND makes use of Router Solicitation (RS)/Router Advertisement (RA) message exchanges similar to classic ND. In 6LoWPAN-ND RA messages may carry additional options for context dissemination and are reduced in size. In addition to RS and RA messages, 6LoWPAN-ND defines two new ICMP packet types: Node Registration (NR), which is sent by a node to an Edge Router to register a binding for an IPv6 address in the Whiteboard, and Node Confirmation (NC) by which an Edge Router replies to the registering node. 

