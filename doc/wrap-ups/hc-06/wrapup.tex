\documentclass[a4paper,11pt,oneside]{article} 
 
\usepackage{parskip}                %% blank lines between paragraphs, no indent 
\usepackage[pdftex]{graphicx}        %% include graphics, preferably pdf 
\usepackage[pdftex]{hyperref}        %% many PDF options can be set here 
\usepackage{graphicx}    % needed for including graphics e.g. EPS, PS 
\usepackage {amssymb} 
\usepackage{hyperref} 

\newenvironment{mylisting}
{\begin{list}{}{\setlength{\leftmargin}{1em}}\item\footnotesize\bfseries}
{\end{list}}

\newenvironment{mytinylisting}
{\begin{list}{}{\setlength{\leftmargin}{1em}}\item\tiny\bfseries}
{\end{list}}

 
\begin{document} 
\section{RFC 4944, Transmission of IPv6 Packets over IEEE 802.15.4 Networks}
RFC4944 \cite{rfc4944} defines a 6LoWPAN adaptation format to carry IPv6 datagrams over such constrained links, taking into account limited
bandwidth, memory, or energy resources that are expected in applications such as wireless sensor networks:
\begin{itemize}
\item stateless header compression for IPv6 datagrams (LOWPAN\_HC1 and LOWPAN\_HC2) to reduce the relatively large IPv6 and UDP headers down to (in the best case) several bytes;
\item Fragmentation header to support the IPv6 minimum MTU requirement \cite{rfc2460};
\item Mesh Addressing header to support Mesh Under sub-IP forwarding;
\end{itemize}

\section{ Internet Draft, Compression Format for IPv6 Datagrams in 6LoWPAN Networks} 
The draft \cite{draft-hc-06} specifies a header compression format that is intended
   to replace the LOWPAN\_HC1 defined in RFC4944 \cite{rfc4944}.

\subsection{Motivation}
\begin{itemize}
\item LOWPAN\_HC1 and LOWPAN\_HC2 are insufficient for most practical uses of 6LoWPAN networks.  LOWPAN\_HC1 is most effective for link-local unicast communication, where IPv6 addresses carry the link-local prefix and an Interface Identifier (IID) directly derived from IEEE 802.15.4 addresses \cite{rfc4291}.  In this case, both addresses may be completely elided. 

\item Routable addresses must be used when communicating with devices
external to the LoWPAN or in a route-over configuration where IP
forwarding occurs within the LoWPAN.  For routable addresses,
LOWPAN\_HC1 requires both IPv6 source and destination addresses to
carry the prefix in-line. 
\end{itemize}

\subsection{New features} 
The draft defined two encoding formats:
\begin{itemize}
\item LOWPAN\_IPHC -- for effective compression of Unique Local, Global, and multicast IPv6
addresses based on shared state within contexts;

\item LOWPAN\_NHC -- for encoding arbitrary headers following the LOWPAN\_IPHC header.
\end{itemize}

\subsection{IPv6 Header Compression} 
The draft \cite{draft-hc-06} introduces LOWPAN\_IPHC compression format for the IPv6 header. LOWPAN\_IPHC assumes the following common case IPv6 header values for 6LoWPAN communication:
\begin{itemize}
\item Version is 6; 
\item Traffic Class and Flow Label are both zero; 
\item Payload Length can be inferred from lower layers from either the 6LoWPAN Fragmentation header or the IEEE 802.15.4 header;
\item Hop Limit will be set to a well-known value by the source; 
\item addresses assigned to 6LoWPAN interfaces will be formed using the link-local prefix or a single routable prefix assigned to the entire 6LoWPAN network; 
\item addresses assigned to 6LoWPAN interfaces are formed with an IID derived directly from either the 64-bit extended or 16-bit short IEEE 802.15.4 addresses.
\end{itemize}

The compression mechanism is adapted for these values of the header fields and  in such scenario the LOWPAN\_IPHC can compress the IPv6 header down to two octets (1-octet dispatch, 1-octet LOWPAN\_IPHC) with link-local communication.

\subsection{LOWPAN\_IPHC Encoding Format}
The IPv6 header encrypted with the LOWPAN\_IPHC encoding format consists of three parts (shown in Figure \ref{fig:lowpaniphc}):
\begin{itemize}
\item Dispatch type -- in case of LOWPAN\_IPHC is 3 bit value $110$;
\item LOWPAN\_IPHC -- encoding that describes how an IPv6 header is compressed;
\item Compressed IPv6 Header -- the IPv6 header fields that are not fully elided, either in a compressed form if the field is partially elided, or litteraly.
\end{itemize}
\begin{figure}[htp]
\centering
\begin{mylisting}
\begin{verbatim}
+----------+-------------+-----------------------+
| Dispatch | LOWPAN_IPHC | Compressed IPv6 Header|
+----------+-------------+-----------------------+
\end{verbatim}
\end{mylisting}
\caption{LOWPAN\_IPHC Header}\label{fig:lowpaniphc}
\end{figure}

The LOWPAN\_IPHC encoding (shown in Figure) \ref{fig:lowpanbe} usually utilizes 13 bits, but may be extended by another octet to support additional contexts. 
\begin{figure}[htp]
\centering
\begin{mylisting}
\begin{verbatim}
  0   1   2   3   4   5   6   7   8   9   0   1   2  
+---+---+---+---+---+---+---+---+---+---+---+---+---+
|  TF   |NH | HLIM  |CID|SAC|  SAM  | M |DAC|  DAM  |
+---+---+---+---+---+---+---+---+---+---+---+---+---+
\end{verbatim}
\end{mylisting}
\caption{LOWPAN\_IPHC Encoding}\label{fig:lowpanbe}
\end{figure}

The encoding supports 4 compression formats of the Traffic Class and Flow Label (TF field). The
NH field specifies whether the IPv6 Next header field is carried in-line or compressed using LOWPAN\_NHC (described in Section \ref{sec:ipnh}). The HLIM encodes the compression type for the Hop Limit field. The CID field indicates
whether an additional 8-bit Context Identifier Extension field immediately follows the DAM field. 

The SAC, SAM, DAC and DAM fields describe how the source and destination addresses are compressed. The LOWPAN\_IPHC encoding supports stateless and stateful context-based compressions. The stateless compression makes use of the link-local prefixes (the value of those bits is the link-local prefix) and link-local addresses (derivable from the corresponding link-layer address). The stateful address compression can be applied to the source and destination IPv6 addresses when they do not statelessly match the source and destination link layer addresses.    
The stateful compression relies on a conceptual context which is shared between the node that compresses a packet and the node(s) that need to expand it. However, the draft does not specify how these shared contexts are established and how the information is maintained within them.

The M field specifies whether the Destination Address is a multicast address or not.


\subsection{IPv6 Next Header Compression} \label{sec:ipnh}
When the NH bit of LOWPAN\_IPHC is set to 1 the 6LoWPAN next header compression (LOWPAN\_NHC) is used. 
This mechanism is intented to replace the LOWPAN\_HC2 compression defined in \cite{rfc4944}. In contrast, LOWPAN\_NHC allows to define comparession formats for arbitrary next headers, whereas LOWPAN\_HC2 can be used only for UDP, TCP, and ICMPv6.

The value of IPv6 Next Header is recovered from the first bits in the LOWPAN\_NHC encoding and the remaining 
bits are specific to the IPv6 Next Header value. Compression formats for different next headers are identified by a
bit-pattern immediately following the LOWPAN\_IPHC compressed header and can have different length.
The draft document defines a set of LOWPAN\_NHC encodings for selected IPv6 Extension Headers (i.e. UDP Header Compression).
\begin{figure}[htp]
\centering
\begin{mylisting}
\begin{verbatim}
+-------------+-----------+-------------+---------------+--------
| LOWPAN_IPHC | In-line   | LOWPAN_NHC  | In-line Next  | Payload
|   Encoding  | IP Fields |   Encoding  | Header Fields |
+-------------+-----------+-------------+---------------+--------
\end{verbatim}
\end{mylisting}
\caption{Typical LOWPAN\_IPHC/LOWPAN\_NHC Header Configuration}\label{fig:lowpaniphc}
\end{figure}


\nocite{rfc2460} 
\nocite{rfc4291} 
\nocite{rfc4944} 
\nocite{draft-hc-06} 
 
\bibliographystyle{plain} 
\bibliography{wrapup} 
\end{document} 
 
