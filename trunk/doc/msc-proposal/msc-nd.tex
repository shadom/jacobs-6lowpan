\chapter{6LoWPAN Neighbor Discovery Protocol}\label{ch:nd}
The Neighbor Discovery (ND) protocol for IPv6 \cite{rfc4861} provides for basic bootstrapping and network operation. Nodes use ND to determine the link-layer addresses for neighbors known to reside on attached links, as well as to find neighboring routers that are willing to forward packets. However, the standard IPv6 Neighbor Discovery has several problems when using it with 6LoWPANs. IPv6 ND heavily uses multicast capabilities, whereas it is very expensive and not desirable in a low-power, lossy wireless network. Moreover, in a route over configuration LoWPAN links are non-transitive and composed of a large amount of overlapping radio ranges, but the classic ND was not designed for such links. Therefore, the classic ND is not suitable for 6LoWPANs. 

The internet draft \cite{draft-nd-07} specifies an optimized neighbor discovery mechanism sufficient for LoWPAN operation. The 6LoWPAN Neighbor Discovery (6LoWPAN-ND) introduces a node registration mechanism optimizing the node-router interface, which requires no flooding and reduces link-local multicast frequency.  The concept of a LoWPAN Whiteboard located at Edge Routers (ERs - routers that connect a LoWPAN to another IP network) is introduced, which allows for Duplicate Address Detection for the entire LoWPAN. The solution supports both Mesh Under and route over configurations for multihop forwarding. This chapter describes the specified 6LoWPAN-ND protocol.

%All nodes in the LoWPAN register with Routers and Edge Routers, though Routers are present only in a route over configuration. Each Router maintains a set of information about nodes that are currently registered through it, called the binding table. All IPv6 addresses in the LoWPAN are stored within a conceptual data structure, Whiteboard, located at Edge Routers (ERs). Nodes send periodic registration messages in order to maintain their bindings in the Router binding tables and Edge Router Whiteboard. 

\section{Bootstrapping and Basic Operation}
Bootstrapping of a LoWPAN node consists of several steps.  At first, a node is required to autoconfigure at least one address, a link-local address, which is derived from the IEEE 64-bit extended MAC.  Therefore,  knowledge of the 64-bit address of a node is enough to derive its link-local address and reach it on the same link. In order to receive Router Advertisements (RAs) from routers, a node joins the all-nodes multicast address and, if the node is a router, the all-routers multicast address. Once the interfaces have been initialized, a node listens for RAs, or broadcasts a Router Solicitation (RS). Upon receipt of the RA, the node forms an optimistic global unique address with stateless address autoconfiguration and chooses one or more default routers (routers are present only in a route over configuration). 

The constructed global address is tentative as long as the binding is not confirmed by the ER. To accomplish this, the node performs initial registration with a link-local ER  or LoWPAN Router by sending a unicast Node Registration (NR) message with a list of all addresses it wants to register. Registering directly with an ER is preferred, although all LoWPAN Routers have the ability to relay NR/NC messages on behalf of a node to the ER. The ER replies with a Node Confirmation(NC), which includes the set of addresses confirmed to be bound to the Whiteboard of the ER. Once the node has received the NC message, it is capable to send packets to any IPv6 address inside or outside the LoWPAN. Each router maintains information about nodes that are currently registered through it, called the binding table. Nodes send periodic registration messages in order to maintain their bindings in the router binding tables and Edge Router Whiteboard. The detection of duplicate addresses (DAD) is also performed as part of the node registration process by the ER using a lookup on the Whiteboard. 

Next-hop determination assumes destinations are non-local and forwarded to a default router except for link-local scope addresses. The information about link-local addresses is collected during the node registration process. Nodes store router link-local addresses in their default router list, whereas routers resolve the link-layer addresses of the attached nodes by looking up on the binding table. 

\section{Message Types}

6LoWPAN-ND makes use of Router Solicitation (RS) and Router Advertisement (RA) message exchanges in a similar way to classic ND. However, 6LoWPAN-ND RA messages may carry additional options for context dissemination and are reduced in size. In addition to RS and RA messages, 6LoWPAN-ND defines two new ICMP packet types, Node Registration (NR) and Node Confirmation (NC), used by a node to register with an ER, and by the ER to confirm the
binding. 

\subsection{Router Solicitation Message}
The format of RS message for 6LoWPAN is identical to the classic RS message \cite{rfc4861}. If a node has only the addresses that are not yet confirmed by an ER, then the IPv6 unspecified address is used as the IPv6 source address. The Source Link-Layer Address Option \cite{rfc4861} used in IPv6 ND is not included in the RSs and the Owner Interface Identifier Option, described in Section \ref{sec:nd.oiio}, is used instead.

\subsection{Router Advertisement Message}
The format of RA message for 6LoWPAN is identical to the classic \cite{rfc4861} RA message. However, the use of a 2-bit  Default Router Preference (Prf) flag \cite{rfc4191} is defined in the context of 6LoWPAN. LoWPAN Edge Routers set Prf to $01$ indicating their high preference, whereas LoWPAN Routers with and without ER availability use the values $00$ and $01$ respectively for normal and low preference. RAs are sent either to all-nodes multicast or to a link-local unicast address as a response to an RS. RA messages use additional 6LoWPAN Information and 6LoWPAN Summary Options described in Section \ref{nd.option.info}.

\subsection{Neighbor Solicitation/Advertisement Message}
Neighbor Solicitation(NS) and Neighbor Advertisement(NA) messages are employed only between ERs on the backbone link in an Extended LoWPAN. Extended LoWPAN (shown in Figure \ref{fig:ext.lowpan})) allows to form a single subnet out of multiple LoWPANs interconnected by a backbone link via Edge Routers. The format of NS/NA messages is identical to the classic NS/NA messages \cite{rfc4861} with additional use of the Owner Interface Identifier Option, described in Section \ref{sec:nd.oiio}. 

\begin{figure}[htp]
\begin{mylisting}
\begin{verbatim}
               +-----+                 +-----+
               |     | Router          |     | Host
               |     |                 |     |
               +-----+                 +-----+
                  |                       |
                  |     Backbone link     |
            +--------------------+------------------+
            |                    |                  |
         +-----+             +-----+             +-----+
         |     | Edge        |     | Edge        |     | Edge
         |     | Router      |     | Router      |     | Router
         +-----+             +-----+             +-----+
            o         o       o   o  o      o        o o
        o o   o  o  o  o  o o   o  o  o  o  o   o  o  o  o
       o  o o  o o   o    o   o  o  o  o     o   o  o  o o
       o   o  o  o     o    o    o  o     o      o  o   o
         o   o o     o          o  o      o    o       o
\end{verbatim}
\end{mylisting}
\caption{Extended LoWPAN}\label{fig:ext.lowpan}
\end{figure}

\subsection{Node Registration/Confirmation Message}
In order to register for the first time, the node sends an NR message to the link-local unicast address of a local ER or LoWPAN Router. Since the node still has only an optimistic address, not yet confirmed by an Edge Router, the IPv6 unspecified address is used as the source address of the NR message. To renew bindings the subsequent NR messages use the link-local IPv6 address of the sender as the source address.

The message format for NR/NC messages is shown in Figure \ref{fig:nr.nc.format}. The values for the Type field will be defined in the future to distinguish Node Registration and Node Confirmation messages. The Status field is used only in NC messages and specifies the result of the registration procedure. A 4-bit unsigned Code provides additional details for the message, e.g., it may indicate whether the NR/NC is sent directly or relayed by a router. The Checksum field carries the ICMP checksum. A unique Transaction ID (TID) is used to match replies and a NC TID corresponds to the NR TID. Its value is incremented upon each new registration. A 1-bit Primary flag (P) indicates whether the ER is primary and may represent the node on the backbone in a Extended LoWPAN. A Router flag indicates the role of the node sending the NR message (0 is used by hosts and 1 by routers).  The Reserved field is unused and initialized to zero. The Binding Lifetime is the amount of time in minutes remaining before the binding of the owner interface identifier expires. The Advertising Lifetime field indicates the amount of time in units of 10 seconds the node will advertise itself to its local router using a NR local refresh. The Owner Nonce field is a 32-bit value generated randomly by the node upon booting.
And the Owner Interface Identifier (OII) is a globally unique identifier for the requesting host's interface.

A NR message includes a 6LoWPAN  Address Option, described in Section \ref{nd.option.address}, for each address the host wants to bind for the interface, and a NC message may carry a 6LoWPAN Information Option, described in Section \ref{nd.option.info}.

\begin{figure}[htp]
\begin{mylisting}
\begin{verbatim}
    0                   1                   2                   3
    0 1 2 3 4 5 6 7 8 9 0 1 2 3 4 5 6 7 8 9 0 1 2 3 4 5 6 7 8 9 0 1
   +-+-+-+-+-+-+-+-+-+-+-+-+-+-+-+-+-+-+-+-+-+-+-+-+-+-+-+-+-+-+-+-+
   |     Type      |Status | Code  |           Checksum            |
   +-+-+-+-+-+-+-+-+-+-+-+-+-+-+-+-+-+-+-+-+-+-+-+-+-+-+-+-+-+-+-+-+
   |      TID      |P|R|                Reserved                   |
   +-+-+-+-+-+-+-+-+-+-+-+-+-+-+-+-+-+-+-+-+-+-+-+-+-+-+-+-+-+-+-+-+
   |        Binding Lifetime       |     Advertising Interval      |
   +-+-+-+-+-+-+-+-+-+-+-+-+-+-+-+-+-+-+-+-+-+-+-+-+-+-+-+-+-+-+-+-+
   |                         Owner Nonce                           |
   +-+-+-+-+-+-+-+-+-+-+-+-+-+-+-+-+-+-+-+-+-+-+-+-+-+-+-+-+-+-+-+-+
   |                                                               |
   +                  Owner Interface Identifier                   +
   |                                                               |
   +-+-+-+-+-+-+-+-+-+-+-+-+-+-+-+-+-+-+-+-+-+-+-+-+-+-+-+-+-+-+-+-+
   |   Registration option(s)...
   +-+-+-+-+-+-+-+-+-+-+-+-+-+
\end{verbatim}
\end{mylisting}
\caption{Node Registration/Confirmation message format}\label{fig:nr.nc.format}
\end{figure}

\subsection{Message Options}
The 6LoWPAN-ND introduces four new message options described in this section.

\subsubsection{6LoWPAN Address Option}\label{nd.option.address}
The 6LoWPAN Address Option (6AO) is used in NR and NC messages to indicate the address which a node wants to register with an ER and to get the result of the registration back. A NR/NC message can include multiple Address Options. The fields of the option allow to indicate the success or failure of the binding in an NC, as well as to compress the carried IPv6 address to different extent. 

\subsubsection{6LoWPAN Information Option}\label{nd.option.info}
The 6LoWPAN Information Option (6IO) is used to carry prefix information. A RA message includes one Information Option for each listed prefix.  The option also allows to disseminate contexts used in 6LoWPAN address compression and identified by a Context Identifier (CID).

\subsubsection{6LoWPAN Summary Option}\label{nd.option.summary}
The 6LoWPAN Summary Option (6SO) associates the current prefix options with a sequence number. It allows prefix options themselves to be sent only periodically in unsolicited RAs, reducing the message size. When the sequence number of this option has a new value, then the prefix information has likely changed, and, in this case, a node requests the prefix information with an RS. An RA sent in response to a unicast RS always includes the full set of prefix information.

\subsubsection{Owner Interface Identifier Option}\label{sec:nd.oiio}
The Owner Interface Identifier Option (OIIO) is used with classic NS and NA messages between ERs over a backbone link to identify the entries within a binding table or whiteboard.

\section{Conceptual Data Structures }
\subsection{LoWPAN Node}
Every node in a LoWPAN keeps a list of prefixes advertised in Router Advertisements. The entries of the list are associated with the sequence numbers last advertised in the 6LoWPAN Summary Option, which allows to detect the changes in prefix information. In order to support stateful address compression, a node holds a list of contexts. As in the case of prefixes, each context entry is associated with the 6LoWPAN Summary Option sequence number. The list of Edge Routers the node is registered with an associated timeout and primary flag is stored within the Edge Router List. 
The Default Router List data structure contains the routers to which packets may be forwarded from the node. 

In addition to the mentioned data structures, a node also keeps a set of conceptual variables for each interface, e.g., the default hop limit and the binding lifetime.

\subsection{LoWPAN Router}
LoWPAN Routers are used only in a route over configuration. In addition to the classic ND conceptual variables defined in Section 6.2.1 of \cite{rfc4861}, routers maintain binding tables with the information about nodes that are currently registered through them. A binding table entry  contains the registered node's OII, link-local IPv6 address and the advertisement interval from the last NR.

\subsection{LoWPAN Edge Router}
Edge Routers implement LoWPAN Router features and extend them with whiteboard registration for LoWPAN Nodes within their subnets. An ER maintains information about every registered node, e.g., the IPv6 address and Owner Interface Identifier of the node. There is no link-layer information stored in the Whiteboard. In an Extended LoWPAN the full registry of all the LoWPAN Nodes in a subnet is distributed between the Edge Routers.


